\chapter{GENEL BİLGİLER VE LİTERATÜR ÖZETİ}

Bu bölüm, tezin asıl gövde kısmının ilk bölümüdür. Tez çalışmasına konu olan problemin arka planı, teknik terimlerin açıklaması ve literatürdeki geçmiş çalışmalar genellikle bu ilk bölümde anlatılır. Makale referansı (\cite{azizoglu2024meld3}), bildiri referansı (\cite{azizoglu2021novel}), tez referansı (\cite{azizoglu2020tersinir}), elektronik kaynak referansı (\cite{azizogluGithub}),  kitap bölümü referansı( \cite{azizoglu2023qr}).

\section{Çalışma Alanının Tanıtımı}
Genel problemin tanıtıldığı ve temel kavramların (`equation`, `figure` vs. formunda) sunulduğu bölümdür.
\begin{equation}
E = mc^2
\label{eq:ornekDenklem}
\end{equation}

\subsection{Alt Konular}
Tez formatını test etmek için bir adet alt başlık. Denklem \ref{eq:ornekDenklem}'de görüldüğü gibi referans verilebilir.

\subsubsection{Derinlere İndikçe}
Gerekirse dördüncü veya beşinci seviye başlıklar da eklenebilir. 

\begin{table}[H]
    \centering
    \caption{Popüler derin öğrenme modellerinin performans ve parametre karşılaştırması}
    \begin{tabular}{llcc}
        \toprule
        \textbf{Model Adı} & \textbf{Yayın Yılı} & \textbf{Parametre Sayısı (Milyon)} & \textbf{Top-1 Doğruluk (\%)} \\
        \midrule
        VGG-16 & 2014 & 138.4 & 71.3 \\
        ResNet-50 & 2015 & 25.6 & 74.9 \\
        Inception-v3 & 2015 & 23.8 & 77.9 \\
        DenseNet-121 & 2017 & 8.0 & 75.0 \\
        EfficientNet-B0 & 2019 & 5.3 & 77.1 \\
        \bottomrule
    \end{tabular}
    \label{tab:ornekTablo}
\end{table}

Tablo \ref{tab:ornekTablo}'da basit bir verinin LaTeX üzerinde profesyonel gösterimi sergilenmektedir. Tabloya veya diğer görsellere bu şekilde numara atayarak referans verebilirsiniz.

\begin{figure}[H]
    \centering
    \includegraphics[width=\textwidth]{ornek_sekil.png}
    \caption{Derin öğrenme mimarilerinin temel yapısını gösteren örnek bir şekil.}
    \label{fig:ornekSekil}
\end{figure}

Yine Şekil \ref{fig:ornekSekil}'de olduğu gibi tezin içindeki herhangi bir resme doğrudan gönderme yapabilirsiniz.