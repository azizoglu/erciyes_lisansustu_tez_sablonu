\textbf{KİŞİSEL BİLGİLER}
\vspace*{-5mm}
\renewcommand{\arraystretch}{1.0}
\begin{table*}[h]
\begin{tabular}{p{4cm}ll}
\textbf{Adı Soyadı} & : & Ad SOYAD \\
\textbf{Uyruğu} & : & Türkiye (T.C.) \\
\textbf{Doğum Tarihi ve Yeri} & : & GG.AA.YYYY, Doğum Yeri \\
\textbf{Medeni Durum} & : & Buraya yazınız \\
\textbf{E-mail} & : & email@erciyes.edu.tr \\ 
\textbf{Yazışma Adresi} & : & Buraya yazınız \\
\end{tabular}
\end{table*}

% =======================================================

\textbf{EĞİTİM}
\vspace*{-2mm}
\begin{table}[htbp]
\centering
\begin{tabularx}{\textwidth}{|l|X|c|}
\hline
\textbf{Derece} & \textbf{Kurum} & \textbf{Mezuniyet Tarihi} \\ \hline
Yüksek Lisans & Üniversite İsmi, Bölüm & YYYY \\ \hline
Lisans & Üniversite İsmi, Bölüm & YYYY \\ \hline
Lise & Lise Adı & YYYY \\ \hline
\end{tabularx}
\end{table}

% =======================================================

\textbf{İŞ DENEYİMLERİ}
\vspace*{-2mm}
\begin{table}[h]
\centering
\begin{tabularx}{\textwidth}{|c|X|l|}
\hline
\textbf{Yıl} & \textbf{Kurum} & \textbf{Görev} \\ \hline
YYYY–Halen & Çalisilan Kurum Adı & Görev Adı \\ \hline
YYYY–YYYY & Çalisilan Kurum Adı & Görev Adı \\ \hline
\end{tabularx}
\end{table}

% =======================================================

\textbf{YABANCI DİL}

İngilizce
\vspace*{5mm}

% =======================================================

\textbf{YAYINLAR}
\vspace*{-7mm}
\begin{enumerate}
\item Soyadı, A., Yıl. Makalenin adı (Sözcüklerin ilk harfi küçük). \textbf{Yayımlandığı derginin açık ve tam adı, Cilt} (Sayı): Sayfa aralığı.
\item Soyadı, A., Yıl. Makalenin adı (Sözcüklerin ilk harfi küçük). \textbf{Yayımlandığı derginin açık ve tam adı, Cilt} (Sayı): Sayfa aralığı.
\end{enumerate}